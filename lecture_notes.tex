\documentclass[12pt,a4paper]{book}
\usepackage[utf8]{inputenc}
\usepackage[T1]{fontenc}
\usepackage{amsmath,amsfonts,amssymb}
\usepackage{graphicx}
\usepackage{float}
\usepackage{booktabs}
\usepackage{longtable}
\usepackage{array}
\usepackage{geometry}
\usepackage{fancyhdr}
\usepackage{hyperref}
\usepackage{listings}
\usepackage{xcolor}
\usepackage{tcolorbox}
\usepackage{enumitem}
\usepackage{tikz}
\usepackage{pgfplots}
\usepackage{subcaption}

% Page geometry
\geometry{
    left=1in,
    right=1in,
    top=1in,
    bottom=1in,
    headheight=15pt  % Fix header height warning
}

% Header and footer
\pagestyle{fancy}
\fancyhf{}
\fancyhead[LE,RO]{\thepage}
\fancyhead[LO]{\nouppercase{\leftmark}}
\fancyhead[RE]{Mathematics and Descriptive Statistics in Excel}

% Hyperref setup
\hypersetup{
    colorlinks=true,
    linkcolor=blue,
    filecolor=magenta,
    urlcolor=cyan,
    pdftitle={Mathematics and Descriptive Statistics in Excel},
    pdfauthor={Frederic Mirindi},
    pdfsubject={Lecture Notes for Undergraduate Students}
}

% Code listings setup
\lstset{
    backgroundcolor=\color{gray!10},
    basicstyle=\ttfamily\small,
    breaklines=true,
    captionpos=b,
    frame=single,
    numbers=left,
    numberstyle=\tiny\color{gray},
    showspaces=false,
    showstringspaces=false,
    tabsize=2
}

% Custom environments
\newtcolorbox{definition}[1]{
    colback=blue!5!white,
    colframe=blue!75!black,
    title=#1
}

\newtcolorbox{example}[1]{
    colback=green!5!white,
    colframe=green!75!black,
    title=#1
}

\newtcolorbox{note}[1]{
    colback=yellow!5!white,
    colframe=orange!75!black,
    title=#1
}

\title{\Huge\textbf{Mathematics and Descriptive Statistics in Excel}\\[0.5cm]
       \Large Lecture Notes for Undergraduate Students}
\author{\textbf{Fr\'{e}d\'{e}ric Mirindi}\\[0.3cm]
        \textit{Lecturer}\\[0.2cm]
        \textit{Booth University College}\\[0.5cm]
        \texttt{frederic.mirindi@boothuc.ca}}
\date{\today}

\begin{document}

\maketitle

\tableofcontents
\listoffigures
\listoftables

\chapter*{Preface}
\addcontentsline{toc}{chapter}{Preface}

These lecture notes are designed for undergraduate students studying mathematics and descriptive statistics with practical applications in Microsoft Excel. The course combines theoretical foundations with hands-on experience, enabling students to understand statistical concepts while developing proficiency in one of the most widely used data analysis tools in business and research.

The notes are structured to provide a comprehensive understanding of mathematical concepts underlying statistical analysis, followed by practical implementation using Excel's built-in functions and features. Each chapter includes theoretical explanations, worked examples, and Excel-based exercises to reinforce learning.

\textbf{Prerequisites:} Basic algebra, introductory calculus (recommended), and familiarity with computer operations.

\textbf{Learning Objectives:}
\begin{itemize}
    \item Understand fundamental mathematical concepts in statistics
    \item Master descriptive statistical measures and their interpretations
    \item Develop proficiency in Excel for statistical analysis
    \item Apply statistical methods to real-world problems
    \item Interpret and communicate statistical results effectively
\end{itemize}

\vfill
\textit{Fr\'{e}d\'{e}ric Mirindi}\\[0.2cm]
Booth University College\\[0.2cm]
Winnipeg, Manitoba, Canada\\[0.2cm]
\today

\newpage

% Chapter 1: Introduction to Mathematics for Statistics
\chapter{Introduction to Mathematics for Statistics}

\section{Overview}

Statistics is fundamentally a mathematical discipline that provides tools for collecting, analyzing, interpreting, and presenting data. This chapter establishes the mathematical foundation necessary for understanding statistical concepts and their implementation in Excel.

\section{Essential Mathematical Concepts}

\subsection{Set Theory and Notation}

\begin{definition}{Set Theory Basics}
A set is a collection of distinct objects. In statistics, we often work with:
\begin{itemize}
    \item Sample space ($S$ or $\Omega$): the set of all possible outcomes
    \item Events ($A$, $B$, etc.): subsets of the sample space
    \item Universal set: the set containing all elements under consideration
\end{itemize}
\end{definition}

\textbf{Set Operations:}
\begin{itemize}
    \item Union: $A \cup B$ (elements in A or B or both)
    \item Intersection: $A \cap B$ (elements in both A and B)
    \item Complement: $A^c$ (elements not in A)
    \item Difference: $A - B$ (elements in A but not in B)
\end{itemize}

\section{Exercises}

\begin{enumerate}
    \item Create an Excel worksheet demonstrating the properties of summation notation.
    \item Use Excel to verify De Morgan's laws: $(A \cup B)^c = A^c \cap B^c$
    \item Calculate probabilities for a standard deck of cards using Excel functions.
\end{enumerate}

% Chapter 2: Data Types and Measurement Scales
\chapter{Data Types and Measurement Scales}

\section{Introduction to Data}

Data is the foundation of statistical analysis. Understanding different types of data and their properties is crucial for selecting appropriate analytical methods and Excel functions.

\section{Classification of Data}

\subsection{Qualitative vs. Quantitative Data}

\begin{definition}{Data Classifications}
\textbf{Qualitative (Categorical) Data:}
\begin{itemize}
    \item Nominal: Categories with no natural order (e.g., colors, gender)
    \item Ordinal: Categories with natural order (e.g., education levels, satisfaction ratings)
\end{itemize}

\textbf{Quantitative (Numerical) Data:}
\begin{itemize}
    \item Discrete: Countable values (e.g., number of students, defects)
    \item Continuous: Measurable values on a continuum (e.g., height, weight, time)
\end{itemize}
\end{definition}

\section{Exercises}

\begin{enumerate}
    \item Create a dataset with examples of all four measurement scales
    \item Build frequency distributions for both categorical and numerical data
    \item Construct histograms with different bin widths and analyze the effect
\end{enumerate}

% Chapter 3: Measures of Central Tendency
\chapter{Measures of Central Tendency}

\section{Introduction}

Measures of central tendency provide single values that represent the ``center'' or ``typical'' value of a dataset. The three primary measures are the mean, median, and mode, each with specific applications and interpretations.

\section{Arithmetic Mean}

\subsection{Population Mean}

\begin{definition}{Population Mean}
For a population of $N$ values:
\[
\mu = \frac{\sum_{i=1}^{N} x_i}{N} = \frac{x_1 + x_2 + \ldots + x_N}{N}
\]
where $\mu$ (mu) represents the population mean.
\end{definition}

\subsection{Sample Mean}

\begin{definition}{Sample Mean}
For a sample of $n$ values:
\[
\bar{x} = \frac{\sum_{i=1}^{n} x_i}{n} = \frac{x_1 + x_2 + \ldots + x_n}{n}
\]
where $\bar{x}$ (x-bar) represents the sample mean.
\end{definition}

\section{Exercises}

\begin{enumerate}
    \item Calculate all three measures of central tendency for a given dataset
    \item Compare mean and median for skewed distributions
    \item Create Excel formulas for weighted averages
    \item Analyze the effect of outliers on different measures
\end{enumerate}

% Chapter 4: Measures of Variability
\chapter{Measures of Variability}

\section{Introduction}

Measures of variability (or dispersion) describe how spread out or scattered the data values are around the central tendency. While measures of central tendency tell us about the ``typical'' value, measures of variability tell us about the consistency or reliability of our data.

\section{Range}

\begin{definition}{Range}
The range is the simplest measure of variability:
\[
\text{Range} = \text{Maximum value} - \text{Minimum value}
\]
\end{definition}

\begin{example}{Excel Range Calculation}
\texttt{=MAX(range) - MIN(range)}
\end{example}

\section{Variance}

\subsection{Population Variance}

\begin{definition}{Population Variance}
\[
\sigma^2 = \frac{\sum_{i=1}^{N} (x_i - \mu)^2}{N}
\]
where $\sigma^2$ (sigma squared) is the population variance.
\end{definition}

\subsection{Sample Variance}

\begin{definition}{Sample Variance}
\[
s^2 = \frac{\sum_{i=1}^{n} (x_i - \bar{x})^2}{n-1}
\]
where $s^2$ is the sample variance and $(n-1)$ provides an unbiased estimator.
\end{definition}

\section{Exercises}

\begin{enumerate}
    \item Calculate all measures of variability for a sample dataset
    \item Compare variability between different groups using coefficient of variation
    \item Apply Chebyshev's theorem and empirical rule to real data
    \item Detect outliers using both z-score and IQR methods
    \item Create Excel formulas for automated outlier detection
\end{enumerate}

% Chapter 5: Probability Distributions
\chapter{Probability Distributions}

\section{Introduction}

Probability distributions describe how probabilities are assigned to the possible values of a random variable. Understanding these distributions is essential for statistical inference, hypothesis testing, and decision-making under uncertainty.

\section{Random Variables}

\begin{definition}{Random Variable}
A random variable is a function that assigns numerical values to the outcomes of a random experiment. Random variables can be:
\begin{itemize}
    \item \textbf{Discrete:} Countable values (e.g., number of defects)
    \item \textbf{Continuous:} Uncountable values on an interval (e.g., weight, time)
\end{itemize}
\end{definition}

\section{Normal Distribution}

\begin{definition}{Normal Distribution}
The normal distribution with parameters $\mu$ and $\sigma$:
\[
f(x) = \frac{1}{\sigma\sqrt{2\pi}} e^{-\frac{1}{2}\left(\frac{x-\mu}{\sigma}\right)^2}
\]
Notation: $X \sim N(\mu, \sigma^2)$
\end{definition}

\section{Exercises}

\begin{enumerate}
    \item Calculate binomial probabilities for quality control scenarios
    \item Use Poisson distribution for arrival rate problems
    \item Apply normal distribution to measurement data
    \item Demonstrate Central Limit Theorem with Excel simulations
    \item Create probability calculators using Excel functions
\end{enumerate}

% Chapter 6: Correlation and Regression Analysis
\chapter{Correlation and Regression Analysis}

\section{Introduction}

Correlation and regression analysis examine relationships between variables. Correlation measures the strength and direction of linear relationships, while regression quantifies these relationships and enables prediction.

\section{Correlation Analysis}

\subsection{Pearson Correlation Coefficient}

\begin{definition}{Pearson Correlation Coefficient}
The sample correlation coefficient:
\[
r_{XY} = \frac{\sum_{i=1}^{n}(x_i - \bar{x})(y_i - \bar{y})}{\sqrt{\sum_{i=1}^{n}(x_i - \bar{x})^2 \sum_{i=1}^{n}(y_i - \bar{y})^2}}
\]
\end{definition}

\section{Simple Linear Regression}

\begin{definition}{Simple Linear Regression Model}
Population model:
\[
Y = \beta_0 + \beta_1 X + \varepsilon
\]

Sample regression line:
\[
\hat{Y} = b_0 + b_1 X
\]
where $\hat{Y}$ is the predicted value.
\end{definition}

\section{Exercises}

\begin{enumerate}
    \item Calculate correlation coefficients for various datasets
    \item Perform simple linear regression analysis
    \item Create and interpret residual plots
    \item Make predictions with confidence intervals
    \item Compare different regression models
\end{enumerate}

% Chapter 7: Advanced Excel Statistical Functions
\chapter{Advanced Excel Statistical Functions}

\section{Introduction}

This chapter covers advanced Excel functions and features that extend beyond basic descriptive statistics. These tools enable sophisticated statistical analysis, hypothesis testing, and data modeling within the Excel environment.

\section{Statistical Distribution Functions}

\begin{example}{Excel Statistical Functions}
\begin{itemize}
    \item \texttt{NORM.DIST(x, mean, std\_dev, cumulative)}: Normal distribution
    \item \texttt{T.DIST(x, deg\_freedom, cumulative)}: t-distribution
    \item \texttt{CHISQ.DIST(x, deg\_freedom, cumulative)}: Chi-square distribution
    \item \texttt{BINOM.DIST(k, n, p, cumulative)}: Binomial distribution
\end{itemize}
\end{example}

\section{Exercises}

\begin{enumerate}
    \item Create a comprehensive statistical analysis dashboard
    \item Implement hypothesis testing procedures
    \item Build Monte Carlo simulation models
    \item Develop automated quality control charts
    \item Design optimization problems using Solver
\end{enumerate}

% Chapter 8: Statistical Projects and Case Studies
\chapter{Statistical Projects and Case Studies}

\section{Introduction}

This chapter presents comprehensive projects and case studies that integrate all concepts covered in previous chapters. These real-world applications demonstrate how to apply mathematical and statistical concepts using Excel in various professional contexts.

\section{Project 1: Quality Control Analysis}

\subsection{Background}

A manufacturing company produces electronic components with a target diameter of 10.00 mm. The production process should maintain this specification with minimal variation.

\subsection{Objectives}

\begin{itemize}
    \item Analyze process capability
    \item Identify trends and patterns
    \item Implement statistical process control
    \item Make recommendations for process improvement
\end{itemize}

\section{Project 2: Sales Forecasting Model}

\subsection{Background}

A retail company wants to develop a forecasting model to predict monthly sales based on various factors including advertising expenditure, seasonal effects, and economic indicators.

\section{Exercises}

\begin{enumerate}
    \item Complete one project from start to finish
    \item Create presentation summarizing findings
    \item Develop Excel template for similar analyses
    \item Compare results with statistical software
    \item Design automated monitoring system
\end{enumerate}

% Appendices
\appendix

\chapter{Excel Function Reference}

\section{Statistical Functions Quick Reference}

\begin{table}[H]
\centering
\caption{Essential Excel Statistical Functions}
\begin{tabular}{@{}lll@{}}
\toprule
\textbf{Function} & \textbf{Purpose} & \textbf{Syntax} \\
\midrule
AVERAGE & Arithmetic mean & \texttt{=AVERAGE(range)} \\
MEDIAN & Middle value & \texttt{=MEDIAN(range)} \\
STDEV.S & Sample standard deviation & \texttt{=STDEV.S(range)} \\
VAR.S & Sample variance & \texttt{=VAR.S(range)} \\
CORREL & Correlation coefficient & \texttt{=CORREL(array1, array2)} \\
SLOPE & Regression slope & \texttt{=SLOPE(known\_y, known\_x)} \\
INTERCEPT & Regression intercept & \texttt{=INTERCEPT(known\_y, known\_x)} \\
NORM.DIST & Normal distribution & \texttt{=NORM.DIST(x, mean, std, cum)} \\
T.DIST & t-distribution & \texttt{=T.DIST(x, df, cum)} \\
BINOM.DIST & Binomial distribution & \texttt{=BINOM.DIST(k, n, p, cum)} \\
\bottomrule
\end{tabular}
\end{table}

\chapter{Formulas and Definitions Summary}

\section{Descriptive Statistics}

\begin{align}
\text{Mean: } \bar{x} &= \frac{\sum_{i=1}^{n} x_i}{n}\\
\text{Variance: } s^2 &= \frac{\sum_{i=1}^{n} (x_i - \bar{x})^2}{n-1}\\
\text{Standard Deviation: } s &= \sqrt{s^2}
\end{align}

\section{Correlation and Regression}

\begin{align}
\text{Correlation: } r &= \frac{\sum(x_i - \bar{x})(y_i - \bar{y})}{\sqrt{\sum(x_i - \bar{x})^2 \sum(y_i - \bar{y})^2}}\\
\text{Slope: } b_1 &= \frac{\sum(x_i - \bar{x})(y_i - \bar{y})}{\sum(x_i - \bar{x})^2}
\end{align}

\chapter{Glossary}

\begin{description}
    \item[ANOVA] Analysis of Variance - statistical technique for comparing means across multiple groups
    \item[Central Limit Theorem] Mathematical theorem stating that sample means approach normality as sample size increases
    \item[Correlation] Statistical measure of linear relationship strength between two variables
    \item[Descriptive Statistics] Methods for summarizing and describing data characteristics
    \item[Regression] Statistical method for modeling relationships between variables
    \item[Standard Error] Standard deviation of sampling distribution
    \item[Variance] Measure of data dispersion; average squared deviation from mean
\end{description}

\end{document}